\documentclass{beamer}

\usepackage[utf8]{inputenc}
\usetheme{default}
\title{Choosing a model: From A(IC) to D(IC)}
\author{Giles Innocent \& Matt Denwood}
\date{COST-Harmony Madrid July 2023}

\begin{document}

\maketitle

\begin{frame}
 \frametitle{Why is one model better than another?}
 \begin{itemize}
  \item Describes the data
  \item Fits the data
  \item As simple as possible (Occam's razor - William of Ockham)
 \end{itemize}
 \begin{center}
  \Huge BUT
 \end{center}
 \begin{itemize}
  \item How good is the fit?
  \item How simple is simple?
 \end{itemize}
\end{frame}

\begin{frame}
 \frametitle{Enter Akaike}
 \begin{itemize}
  \item Akaike information criterion (AIC) is an estimator of prediction error and thereby relative quality of statistical models for a given set of data. (Wikipaedia)
  \item Based on information theory.
  \item How much information is in the model?
  \begin{itemize}
   \item depends on the number of parameters estimated in the model
   \item needs a coefficient to weight it equivalnt to the data
   \item \emph{usually} $2k$
  \end{itemize}

  \item How much information is in the residuals?
  \begin{itemize}
   \item sometimes referred to as ``how surprised are we?''
   \item if the model fits perfectly then there are no residuals and no information in the data beyond the model.
   \item information theory indicates that we should use the deviance, i.e. $-2*log(\hat L)$
  \end{itemize}
 \end{itemize}
\end{frame}

\begin{frame}
 \frametitle{Bayesian MCMC}
 \begin{itemize}
  \item The degrees of freedom -- constraints -- in a Bayesian model are difficult to define.
  \item Probablistic constraints, e.g. informative priors, do not have integer degrees of freedom
  \item Spiegelhalter \emph{et al.} (2002) proposed the effective number of estimated parameters in a model $P_D$: a non-integer measure.
 \end{itemize}
\end{frame}

\begin{frame}
 \frametitle{Issues}
 \begin{itemize}
  \item There are multiple ways of calculating $P_D$.
  \item Using at least one method $P_D$ is always less than k-1 (df in the data) \emph{even for over specified models.}
  \item Using DIC alone may lead one to believe in a poorly fittng model. It is better to look at the posterior predictive P-values as well
 \end{itemize}

\end{frame}

\end{document}
